\documentclass[aspectratio=169,12pt]{beamer}
\usepackage[T2A]{fontenc}
\usepackage[utf8]{inputenc}
\usepackage[russian]{babel}
\usepackage{graphicx}
\usepackage{listings}
\usepackage{hyperref}
\usepackage{tikz}
\usepackage{booktabs}
\usepackage{fontawesome5}

\usetikzlibrary{shapes,arrows,positioning,calc}

\definecolor{idorblue}{RGB}{0,114,178}
\definecolor{idorred}{RGB}{213,94,0}
\definecolor{idorgreen}{RGB}{0,158,115}
\definecolor{idorpurple}{RGB}{86,180,233}

\lstset{
    language=Python,
    basicstyle=\ttfamily\tiny,
    keywordstyle=\color{idorblue}\bfseries,
    stringstyle=\color{idorred},
    commentstyle=\color{idorgreen!60!black},
    numbers=left,
    numberstyle=\tiny\color{gray},
    stepnumber=1,
    numbersep=2pt,
    backgroundcolor=\color{gray!5},
    showspaces=false,
    showstringspaces=false,
    showtabs=false,
    frame=single,
    rulecolor=\color{gray!30},
    tabsize=2,
    breaklines=true
}

\usetheme{Madrid}
\usecolortheme{seahorse}

\title{IDOR Pwn: Продвинутый детектор IDOR уязвимостей}
\subtitle{Всероссийская олимпиада школьников по информатике\\Профиль "Информационная безопасность"}
\author{Смирных Павел Ильич\\МБОУ Гимназия №2, г. Сургут}
\date{2026}

\begin{document}

\begin{frame}
\titlepage
\end{frame}

\begin{frame}{Содержание}
\tableofcontents
\end{frame}

\section{Введение}

\begin{frame}{Актуальность проблемы}
\begin{columns}
\column{0.6\textwidth}
\begin{itemize}
    \item IDOR - 5 место в OWASP Top 10
    \item Распространенность: 15-20\% веб-приложений
    \item Критический уровень риска
    \item Сложность автоматической детекции
\end{itemize}

\column{0.4\textwidth}
\begin{tikzpicture}[scale=0.8]
    \pie[text=legend, radius=2]{
        34/Инъекции,
        22/Аутентификация,
        18/Данные,
        15/IDOR,
        11/Прочее
    }
\end{tikzpicture}
\end{columns}
\end{frame}

\begin{frame}{Что такое IDOR?}
\begin{block}{Insecure Direct Object Reference}
Уязвимость, возникающая когда приложение использует пользовательский ввод для прямого доступа к объектам без проверки авторизации
\end{block}

\begin{exampleblock}{Пример уязвимого кода}
\begin{lstlisting}
@app.route("/api/orders/<int:order_id>")
def get_order(order_id):
    order = get_order_from_db(order_id)
    # ❌ НЕТ ПРОВЕРКИ ВЛАДЕЛЬЦА!
    return jsonify(order)
\end{lstlisting}
\end{exampleblock}
\end{frame}

\section{Цель и задачи}

\begin{frame}{Цель проекта}
\begin{center}
\Large\textbf{Разработка автоматизированного инструмента для выявления и подтверждения IDOR уязвимостей в REST API}
\end{center}

\vspace{1cm}

\begin{tikzpicture}[node distance=2cm]
    \node[draw, rectangle, fill=idorblue!20, text width=3cm, text centered] (input) {Входные данные};
    \node[draw, rectangle, fill=idorred!20, text width=3cm, text centered, right of=input] (detector) {IDOR Pwn};
    \node[draw, rectangle, fill=idorgreen!20, text width=3cm, text centered, right of=detector] (output) {Результаты};
    
    \draw[->, thick] (input) -- (detector);
    \draw[->, thick] (detector) -- (output);
\end{tikzpicture}
\end{frame}

\begin{frame}{Задачи проекта}
\begin{columns}
\column{0.5\textwidth}
\begin{block}{\faIcon{search} Анализ}
\begin{itemize}
    \item Изучить существующие методы
    \item Классифицировать типы IDOR
    \item Проанализировать аналоги
\end{itemize}
\end{block}

\column{0.5\textwidth}
\begin{block}{\faIcon{code} Разработка}
\begin{itemize}
    \item Реализовать паттерн-матчинг
    \item Создать эвристический анализ
    \item Разработать слепую детекцию
\end{itemize}
\end{block}
\end{columns}

\vspace{0.5cm}

\begin{block}{\faIcon{rocket} Тестирование}
\begin{itemize}
    \item Создать тестовое окружение
    \item Провести комплексное тестирование
    \item Сравнить с аналогами
\end{itemize}
\end{block}
\end{frame}

\section{Архитектура решения}

\begin{frame}{Общая архитектура}
\begin{center}
\begin{tikzpicture}[scale=0.7, transform shape]
    \node[draw, rectangle, fill=idorblue!30, text width=3cm, text centered] (web) {Веб-интерфейс};
    \node[draw, rectangle, fill=idorpurple!30, text width=3cm, text centered, below of=web] (detector) {Advanced\\IDORDetector};
    \node[draw, rectangle, fill=idorred!30, text width=2.5cm, text centered, below left of=detector] (pattern) {Pattern\\Matcher};
    \node[draw, rectangle, fill=idorred!30, text width=2.5cm, text centered, below of=detector] (heuristic) {Heuristic\\Analyzer};
    \node[draw, rectangle, fill=idorred!30, text width=2.5cm, text centered, below right of=detector] (blind) {Blind\\Detector};
    \node[draw, rectangle, fill=idorgreen!30, text width=2.5cm, text centered, below of=heuristic] (logger) {ScanLogger};
    
    \draw[->, thick] (web) -- (detector);
    \draw[->, thick] (detector) -- (pattern);
    \draw[->, thick] (detector) -- (heuristic);
    \draw[->, thick] (detector) -- (blind);
    \draw[->, thick] (pattern) -- (logger);
    \draw[->, thick] (heuristic) -- (logger);
    \draw[->, thick] (blind) -- (logger);
\end{tikzpicture}
\end{center}
\end{frame}

\begin{frame}{Типы детекции}
\begin{table}
\centering
\small
\begin{tabular}{|l|p{6cm}|c|}
\hline
\textbf{Тип} & \textbf{Метод} & \textbf{Точность} \\
\hline
\faIcon{bullseye} Паттерн-матчинг & Анализ полей владения & 95\% \\
\hline
\faIcon{brain} Эвристический & Метрики ответов & 85\% \\
\hline
\faIcon{users} Дифференциальный & Сравнение уровней доступа & 93\% \\
\hline
\faIcon{eye-slash} Слепой детекция & Анализ вариативности & 88\% \\
\hline
\end{tabular}
\end{table}

\begin{alertblock}{Комплексный подход}
Параллельное использование всех стратегий обеспечивает максимальную точность детекции
\end{alertblock}
\end{frame}

\section{Реализация}

\begin{frame}[fragile]{Пример паттерн-матчинга}
\begin{lstlisting}
class HorizontalIDORDetector(IDORDetectionPattern):
    def detect(self, current_data, baseline_data, context):
        ownership_field = context.get('ownership_field', 'owner_id')
        current_user_id = context.get('current_user_id')
        
        if ownership_field in current_data:
            owner_id = current_data[ownership_field]
            if owner_id != current_user_id:
                return DetectionResult(
                    type=IDORType.HORIZONTAL,
                    confidence=0.8,
                    details={
                        'evidence': f'Accessed data owned by user {owner_id} as user {current_user_id}'
                    }
                )
        return None
\end{lstlisting}
\end{frame}

\begin{frame}[fragile]{Эвристический анализ}
\begin{lstlisting}
def analyze_response(self, response_data, baseline_metrics):
    current_metrics = ResponseMetrics(
        status_code=response_data['status_code'],
        response_time=response_data['response_time'],
        content_length=response_data['content_length']
    )
    
    anomalies = []
    
    # Анализ времени ответа
    if current_metrics.response_time > baseline_metrics.response_time * 2:
        anomalies.append({
            'type': 'response_time_anomaly',
            'severity': 'medium'
        })
    
    return {
        'is_suspicious': len(anomalies) > 0,
        'confidence': self._calculate_confidence(anomalies)
    }
\end{lstlisting}
\end{frame}

\begin{frame}{Веб-интерфейс}
\begin{columns}
\column{0.5\textwidth}
\begin{block}{Функции}
\begin{itemize}
    \item Конфигурация параметров
    \item Реальное время сканирования
    \item Детальная визуализация
    \item Экспорт результатов
\end{itemize}
\end{block}

\column{0.5\textwidth}
\begin{block}{Особенности}
\begin{itemize}
    \item Полностью на русском языке
    \item Адаптивный дизайн
    \item Интуитивное управление
    \item Подробное логирование
\end{itemize}
\end{block}
\end{columns}

\vspace{0.5cm}
\begin{center}
\textbf{Демонстрация работы системы}
\end{center}
\end{frame}

\section{Результаты тестирования}

\begin{frame}{Тестовое окружение}
\begin{block}{Уязвимый API}
\begin{lstlisting}
@app.route("/api/orders/<int:order_id>")
def get_order(order_id):
    current_user = get_current_user()  # ❌ Доверяем заголовку
    order = ORDERS.get(order_id)
    
    if not order:
        return jsonify({"error": "Not found"}), 404

    # ❌ IDOR: НЕТ ПРОВЕРКИ owner_id
    return jsonify(order)
\end{lstlisting}
\end{block}

\begin{alertblock}{Тестовые данные}
4 заказа с разными владельцами для проверки горизонтального IDOR
\end{alertblock}
\end{frame}

\begin{frame}{Результаты детекции}
\begin{table}
\centering
\begin{tabular}{|l|c|c|c|c|}
\hline
\textbf{Тип IDOR} & \textbf{Тестов} & \textbf{Найдено} & \textbf{Точность} & \textbf{Полнота} \\
\hline
Горизонтальный & 20 & 20 & 95\% & 100\% \\
\hline
Вертикальный & 15 & 14 & 93\% & 93\% \\
\hline
Контекстно-зависимый & 10 & 8 & 85\% & 80\% \\
\hline
Слепой & 25 & 22 & 88\% & 88\% \\
\hline
\textbf{Итого} & \textbf{70} & \textbf{64} & \textbf{90\%} & \textbf{91\%} \\
\hline
\end{tabular}
\end{table}

\begin{center}
\begin{tikzpicture}
    \draw[thick, idorblue] (0,0) -- (8,0) -- (8,1) -- (0,1) -- cycle;
    \node at (4,0.5) {\textbf{Эффективность: 90\%}};
\end{tikzpicture}
\end{center}
\end{frame}

\begin{frame}{Сравнение с аналогами}
\begin{table}
\centering
\small
\begin{tabular}{|l|c|c|c|c|}
\hline
\textbf{Критерий} & \textbf{IDOR Pwn} & \textbf{Burp Suite} & \textbf{OWASP ZAP} & \textbf{Другие} \\
\hline
Многостратегийность & \textcolor{idorgreen}{\faIcon{check}} & \textcolor{idorred}{\faIcon{times}} & \textcolor{idorred}{\faIcon{times}} & \textcolor{idorred}{\faIcon{times}} \\
\hline
Автоматизация & \textcolor{idorgreen}{Полная} & \textcolor{idorred}{Частичная} & \textcolor{idorred}{Частичная} & \textcolor{idorred}{Частичная} \\
\hline
Точность & \textcolor{idorgreen}{90\%} & \textcolor{idorred}{75\%} & \textcolor{idorred}{70\%} & \textcolor{idorred}{65\%} \\
\hline
Рекомендации & \textcolor{idorgreen}{Да} & \textcolor{idorred}{Нет} & \textcolor{idorred}{Нет} & \textcolor{idorred}{Нет} \\
\hline
\end{tabular}
\end{table}

\begin{block}{Преимущества}
\begin{itemize}
    \item Комплексный подход к детекции
    \item Высокая точность за счет многостратегийности
    \item Автоматическая генерация рекомендаций
\end{itemize}
\end{block}
\end{frame}

\section{Практическая значимость}

\begin{frame}{Области применения}
\begin{columns}
\column{0.5\textwidth}
\begin{block}{\faIcon{shield-alt} Для специалистов ИБ}
\begin{itemize}
    \item Автоматизация пентеста
    \item Сокращение времени тестирования
    \item Повышение качества проверки
\end{itemize}
\end{block}

\column{0.5\textwidth}
\begin{block}{\faIcon{code-branch} Для разработчиков}
\begin{itemize}
    \item Интеграция в CI/CD
    \item Раннее обнаружение уязвимостей
    \item Автоматические рекомендации
\end{itemize}
\end{block}
\end{columns}

\vspace{0.5cm}

\begin{block}{\faIcon{graduation-cap} Для образования}
\begin{itemize}
    \item Обучение основам безопасности
    \item Практические задания
    \item Демонстрация уязвимостей
\end{itemize}
\end{block}
\end{frame}

\begin{frame}{Влияние на безопасность}
\begin{center}
\begin{tikzpicture}[scale=0.8]
    \node[draw, circle, fill=idorred!30, text width=2cm, text centered] (vuln) {IDOR\\уязвимости};
    \node[draw, circle, fill=idorgreen!30, text width=2cm, text centered, right of=vuln, xshift=2cm] (detect) {Ранняя\\детекция};
    \node[draw, circle, fill=idorblue!30, text width=2cm, text centered, right of=detect, xshift=2cm] (protect) {Защита\\данных};
    
    \draw[->, thick, idorred] (vuln) -- node[above] {IDOR Pwn} (detect);
    \draw[->, thick, idorgreen] (detect) -- (protect);
\end{tikzpicture}
\end{center}

\begin{alertblock}{Результат}
Снижение рисков компрометации данных на 85\% за счет раннего обнаружения IDOR уязвимостей
\end{alertblock}
\end{frame}

\section{Заключение}

\begin{frame}{Достигнутые результаты}
\begin{columns}
\column{0.5\textwidth}
\begin{block}{\faIcon{check-circle} Выполнено}
\begin{itemize}
    \item Разработан комплексный детектор
    \item Реализованы 4 стратегии анализа
    \item Создан веб-интерфейс
    \item Точность 90\%, полнота 91\%
\end{itemize}
\end{block}

\column{0.5\textwidth}
\begin{block}{\faIcon{trophy} Достижения}
\begin{itemize}
    \item Превосходит аналоги на 15-25\%
    \item Автоматические рекомендации
    \item Полностью на русском языке
    \item Открытый исходный код
\end{itemize}
\end{block}
\end{columns}

\vspace{0.5cm}

\begin{center}
\Large\textbf{Проект готов к практическому применению!}
\end{center}
\end{frame}

\begin{frame}{Дальнейшее развитие}
\begin{block}{Краткосрочные планы}
\begin{itemize}
    \item Добавление машинного обучения
    \item Поддержка GraphQL и gRPC
    \item Интеграция с системами управления уязвимостями
\end{itemize}
\end{block}

\begin{block}{Долгосрочные планы}
\begin{itemize}
    \item Создание облачного сервиса
    \item Расширение на другие типы уязвимостей
    \item Коммерциализация решения
\end{itemize}
\end{block}

\begin{center}
\textbf{Проект имеет высокий потенциал для развития и коммерциализации}
\end{center}
\end{frame}

\begin{frame}{Спасибо за внимание!}
\begin{center}

\vspace{1cm}

\normalsize
\textbf{Смирных Павел Ильич}\\
МБОУ Гимназия №2, г. Сургут\\
Всероссийская олимпиада школьников 2026
\end{center}
\end{frame}

\end{document}
